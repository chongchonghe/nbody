% Downloaded from http://www.dfcd.net/articles/latex/latex.html
% Modified by CC He.
% Notes on how to use this header is on Dropbox/Latex_Templates/LaTeX_for_Physicists/README.md
% ***********************************************************
% ******************* PHYSICS HEADER ************************
% ***********************************************************
% Version 2

\documentclass[12pt]{article}
\usepackage{setspace}
\usepackage{amsmath} % AMS Math Package
%\usepackage{fontspec,unicode-math}
%\usepackage{amsthm} % Theorem Formatting
\usepackage{amssymb}	% Math symbols such as \mathbb
\usepackage{units}
\usepackage{listings}
\usepackage{hyperref}
\usepackage{graphicx} % Allows for eps images
\usepackage{multicol} % Allows for multiple columns
\usepackage[dvips,letterpaper,margin=0.75in,bottom=0.5in]{geometry}
\usepackage{natbib}
\usepackage{float}

% Define ads acronyms
\newcommand{\apj}{ApJ}
\newcommand{\apjl}{ApJL}
\newcommand{\apjs}{ApJS}
\newcommand{\aap}{A\&A}
\newcommand{\araa}{ARA\&A}
\newcommand{\ssr}{SSR}
\newcommand{\mnras}{MNRAS}


 % Sets margins and page size
\pagestyle{empty} % Removes page numbers
%\pagestyle{fancy}
\makeatletter % Need for anything that contains an @ command 

\newcommand{\code}{\lstinline}
\renewcommand{\tt}{\texttt}
\newcommand{\m}{\textrm}
\renewcommand{\maketitle} % Redefine maketitle to conserve space
{ \begingroup \vskip 10pt \begin{center} \large {\bf \@title}
	\vskip 10pt \large \@author \hskip 20pt \@date \end{center}
  \vskip 10pt \endgroup \setcounter{footnote}{0} }
\makeatother % End of region containing @ commands
\renewcommand{\labelenumi}{(\alph{enumi})} % Use letters for enumerate
% \DeclareMathOperator{\Sample}{Sample}
\let\vaccent=\v % rename builtin command \v{} to \vaccent{}
\renewcommand{\v}[1]{\ensuremath{\mathbf{#1}}} % for vectors
\newcommand{\gv}[1]{\ensuremath{\mbox{\boldmath$ #1 $}}} 
% for vectors of Greek letters
\newcommand{\uv}[1]{\ensuremath{\mathbf{\hat{#1}}}} % for unit vector
\newcommand{\abs}[1]{\left| #1 \right|} % for absolute value
\newcommand{\avg}[1]{\left< #1 \right>} % for average
%\let\underdot=\d % rename builtin command \d{} to \underdot{}
\renewcommand{\d}[1]{\mathrm{d}#1} % for differential
\newcommand{\dd}[2]{\frac{\d{#1}}{\d{#2}}} % for derivatives
\newcommand{\ddd}[2]{\frac{\mathrm{d}^2 #1}{\d{#2^2}}} % for double 
%derivatives
\newcommand{\pd}[2]{\frac{\partial #1}{\partial #2}} 
% for partial derivatives
\newcommand{\pdd}[2]{\frac{\partial^2 #1}{\partial #2^2}} 
% for double partial derivatives
\newcommand{\pdc}[3]{\left( \frac{\partial #1}{\partial #2}
 \right)_{#3}} % for thermodynamic partial derivatives
\newcommand{\ket}[1]{\left| #1 \right>} % for Dirac bras
\newcommand{\bra}[1]{\left< #1 \right|} % for Dirac kets
\newcommand{\braket}[2]{\left< #1 \vphantom{#2} \right|
 \left. #2 \vphantom{#1} \right>} % for Dirac brackets
\newcommand{\matrixel}[3]{\left< #1 \vphantom{#2#3} \right|
 #2 \left| #3 \vphantom{#1#2} \right>} % for Dirac matrix elements
\newcommand{\grad}[1]{\gv{\nabla} #1} % for gradient
\let\divsymb=\div % rename builtin command \div to \divsymb
\renewcommand{\div}[1]{\gv{\nabla} \cdot #1} % for divergence
\newcommand{\curl}[1]{\gv{\nabla} \times #1} % for curl
\let\baraccent=\= % rename builtin command \= to \baraccent
\renewcommand{\=}[1]{\stackrel{#1}{=}} % for putting numbers above =
\newtheorem{prop}{Proposition}
\newtheorem{thm}{Theorem}[section]
\newtheorem{lem}[thm]{Lemma}
%\theoremstyle{definition}
\newtheorem{dfn}{Definition}
%\theoremstyle{remark}
%\newtheorem*{rmk}{Remark}


% Define \stretchint, a macro to input large integrals.
\usepackage{scalerel}[2016-12-29]
\def\stretchint#1{\vcenter{\hbox{\stretchto[440]{\displaystyle\int}{#1}}}}
\def\scaleint#1{\vcenter{\hbox{\scaleto[3ex]{\displaystyle\int}{#1}}}}
\def\bs{\mkern-12mu}


\begin{document}

\title{ASTR615 HW\#4}
\author{Group 4: ChongChong He \& }
\date{\today}
\maketitle

\section*{Problem 2}
We perform a simulation of a cluster with Kroupa IMF.
\subsection*{Mass distribution}
We implement in all of our simulations the Kroupa IMF:
\begin{equation}
\phi(m) \propto 
\begin{cases}
m^{-1.3} \; &(0.08M_\odot < m < 0.5M_\odot) \\
0.5 \, m^{-2.3} \; &(0.5M_\odot<m<100M_\odot)
\end{cases}
\end{equation}
after doing transformation we get
\begin{equation}
m=
	\begin{cases}
	-\cfrac{0.566179}{\sqrt[3]{1.7987\, - x} \left(x^3-5.39611 
	x^2+9.70599 x-5.81939\right)} & (0<x<0.760707)\\
	\cfrac{0.166558}{(1.00024\, -x)^{10/13}} & (0.760707<x<1)
	\end{cases}
\end{equation}

\subsection*{Initial setup}
The cluster is a specially uniformly distributed sphere with a radius of 1. The inital 
velocities are from a  Gaussian distribution with a dispersion correspondent to a virial ratio 
of $ \alpha \sim 0.4 $. A virial ratio $ \alpha < 0.5 $ implies the system is bounded. The 
velocity dispersion crossing time is $ t \approx 0.08 $, so we use a step size of 0.001, i.e. 
80 steps per course time.

\subsection{Video}
The video of this specific setup is \textit{simulations/cluster03.mp4}.

%../pp-nbody/a.out cluster03.txt cluster03 1000 0.01 0.001 200 1


%\subsection*{Morphology}
%We use disk galaxies in our simulation. Each disk galaxy has two components: a 
%thin disk and a thick disk. Each component has a density profile $ \rho(r, h) = 
%\rho_0 \, e^{-r/r_{\rm H}} \, e^{-h/h_{\rm H}} $. $ r $ and $ h $ are given by the 
%solution to the equation
%\begin{equation}\label{key}
%	x = 1 - (1 + \frac{r}{r_{\rm H}}) \, e^{-\frac{r}{r_{\rm H}}}
%\end{equation}
%where $ x $ is a uniform random number between 0 and 1.
%
%\subsection*{Disk Rotation}
%We use a uniform angular velocity $ \omega $.

%\subsection*{Setups}
%\begin{table}[h]
%	\centering
%	\begin{tabular}{ccccc}
%%		\hline{}
%		& $M/M_\odot$ & $r_{\rm H}$/kpc & $h_{\rm H} $/kpc & $\omega/?$ \\
%		\hline
%		thin disk & $5 \times 10^{10}$ & 3.5 & 0.3 & 1? \\
%		thick disk & $ 0.75 \times 10^{10} $ & 3.5 & 1.0 & 1? \\
%%		\hline
%	\end{tabular}
%	\caption{Parameters and errors from Lorentzian and Gaussian fits.}
%	\label{fit params}
%\end{table}
%
%\subsection{Units}
%[L] = kpc, [M] = $ 10^9 M_\odot $, [T] = 14.91 Myr.


\end{document}