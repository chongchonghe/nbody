% Downloaded from http://www.dfcd.net/articles/latex/latex.html
% Modified by CC He.
% Notes on how to use this header is on Dropbox/Latex_Templates/LaTeX_for_Physicists/README.md
% ***********************************************************
% ******************* PHYSICS HEADER ************************
% ***********************************************************
% Version 2

\documentclass[12pt]{article}
\usepackage{setspace}
\usepackage{amsmath} % AMS Math Package
%\usepackage{fontspec,unicode-math}
%\usepackage{amsthm} % Theorem Formatting
\usepackage{amssymb}	% Math symbols such as \mathbb
\usepackage{units}
\usepackage{listings}
\usepackage{hyperref}
\usepackage{graphicx} % Allows for eps images
\usepackage{multicol} % Allows for multiple columns
\usepackage[dvips,letterpaper,margin=0.75in,bottom=0.5in]{geometry}
\usepackage{natbib}
\usepackage{float}

% Define ads acronyms
\newcommand{\apj}{ApJ}
\newcommand{\apjl}{ApJL}
\newcommand{\apjs}{ApJS}
\newcommand{\aap}{A\&A}
\newcommand{\araa}{ARA\&A}
\newcommand{\ssr}{SSR}
\newcommand{\mnras}{MNRAS}


 % Sets margins and page size
\pagestyle{empty} % Removes page numbers
%\pagestyle{fancy}
\makeatletter % Need for anything that contains an @ command 

\newcommand{\code}{\lstinline}
\renewcommand{\tt}{\texttt}
\newcommand{\m}{\textrm}
\renewcommand{\maketitle} % Redefine maketitle to conserve space
{ \begingroup \vskip 10pt \begin{center} \large {\bf \@title}
	\vskip 10pt \large \@author \hskip 20pt \@date \end{center}
  \vskip 10pt \endgroup \setcounter{footnote}{0} }
\makeatother % End of region containing @ commands
\renewcommand{\labelenumi}{(\alph{enumi})} % Use letters for enumerate
% \DeclareMathOperator{\Sample}{Sample}
\let\vaccent=\v % rename builtin command \v{} to \vaccent{}
\renewcommand{\v}[1]{\ensuremath{\mathbf{#1}}} % for vectors
\newcommand{\gv}[1]{\ensuremath{\mbox{\boldmath$ #1 $}}} 
% for vectors of Greek letters
\newcommand{\uv}[1]{\ensuremath{\mathbf{\hat{#1}}}} % for unit vector
\newcommand{\abs}[1]{\left| #1 \right|} % for absolute value
\newcommand{\avg}[1]{\left< #1 \right>} % for average
%\let\underdot=\d % rename builtin command \d{} to \underdot{}
\renewcommand{\d}[1]{\mathrm{d}#1} % for differential
\newcommand{\dd}[2]{\frac{\d{#1}}{\d{#2}}} % for derivatives
\newcommand{\ddd}[2]{\frac{\mathrm{d}^2 #1}{\d{#2^2}}} % for double 
%derivatives
\newcommand{\pd}[2]{\frac{\partial #1}{\partial #2}} 
% for partial derivatives
\newcommand{\pdd}[2]{\frac{\partial^2 #1}{\partial #2^2}} 
% for double partial derivatives
\newcommand{\pdc}[3]{\left( \frac{\partial #1}{\partial #2}
 \right)_{#3}} % for thermodynamic partial derivatives
\newcommand{\ket}[1]{\left| #1 \right>} % for Dirac bras
\newcommand{\bra}[1]{\left< #1 \right|} % for Dirac kets
\newcommand{\braket}[2]{\left< #1 \vphantom{#2} \right|
 \left. #2 \vphantom{#1} \right>} % for Dirac brackets
\newcommand{\matrixel}[3]{\left< #1 \vphantom{#2#3} \right|
 #2 \left| #3 \vphantom{#1#2} \right>} % for Dirac matrix elements
\newcommand{\grad}[1]{\gv{\nabla} #1} % for gradient
\let\divsymb=\div % rename builtin command \div to \divsymb
\renewcommand{\div}[1]{\gv{\nabla} \cdot #1} % for divergence
\newcommand{\curl}[1]{\gv{\nabla} \times #1} % for curl
\let\baraccent=\= % rename builtin command \= to \baraccent
\renewcommand{\=}[1]{\stackrel{#1}{=}} % for putting numbers above =
\newtheorem{prop}{Proposition}
\newtheorem{thm}{Theorem}[section]
\newtheorem{lem}[thm]{Lemma}
%\theoremstyle{definition}
\newtheorem{dfn}{Definition}
%\theoremstyle{remark}
%\newtheorem*{rmk}{Remark}


% Define \stretchint, a macro to input large integrals.
\usepackage{scalerel}[2016-12-29]
\def\stretchint#1{\vcenter{\hbox{\stretchto[440]{\displaystyle\int}{#1}}}}
\def\scaleint#1{\vcenter{\hbox{\scaleto[3ex]{\displaystyle\int}{#1}}}}
\def\bs{\mkern-12mu}


\begin{document}

\title{ASTR615 HW\#4}
\author{Group 4: ChongChong He \& }
\date{\today}
\maketitle

\section*{Problem 2: Simulations of Globular Clusters}
We perform a series of simulations of globular clusters.
\subsection*{Setups}

\subsubsection*{Units and Time Scales}
The units are: $ [T] = 2.5395 \times 10^{15}\,$s $\approx 80\,$Myr, 
$ [L] = 3.08 \times 10^{18} \, $cm $ \approx 1 \, $pc, $ [M] = 6.7925 \times 10^{31} \, $g $ 
\approx 0.03416 M_\odot $. In this setup the gravitational constant is unity. 

Suppose a cluster of stars from the results of \textit{RAMSES} simulation which has $ \sim 
$ 1000 stars with Salpeter IMF in a box of length $ L = 50 $. If this stellar system is 
virialized, $ \sigma^2 = M_{\rm tot} / R $, from which we obtain $ \sigma \approx 
27 $. Here I use a mean mass of 0.638 $ M_\odot $ = 20 (code unit) from Salpeter IMF and 
$ R = 25 $. 
Then, the crossing time of dispersion velocity is $ t_{\sigma} = L / \sigma \approx 2 $.
If we choose a step size $ \Delta t  = t_\sigma / 50 = 0.04 $, i.e. 50 steps in a course time, 
then with 5000 steps we are able to simulate 100 course time, which is equal to 16 Gyr. 
This can be done with a PP n-body code on my laptop.

\subsubsection*{Mass distribution}
We implement in all of our simulations the Kroupa IMF:
\begin{equation}
\phi(m) \propto 
\begin{cases}
m^{-1.3} \; &(0.08M_\odot < m < 0.5M_\odot) \\
0.5 \, m^{-2.3} \; &(0.5M_\odot<m<100M_\odot)
\end{cases}
\end{equation}
after doing transformation we get
\begin{equation}
m=
	\begin{cases}
	-\cfrac{0.566179}{\sqrt[3]{1.7987\, - x} \left(x^3-5.39611 
	x^2+9.70599 x-5.81939\right)} & (0<x<0.760707)\\
	\cfrac{0.166558}{(1.00024\, -x)^{10/13}} & (0.760707<x<1)
	\end{cases}
\end{equation}
where $ x $ is uniform random numbers between 0 and 1.

\subsubsection*{Spacial and velocity distribution}
The cluster is a specially uniformly distributed sphere with a radius of 1. The initial 
velocities are from a  Gaussian distribution with a dispersion correspondent to a virial ratio 
of $ \alpha \sim 0.4 $. A virial ratio $ \alpha < 0.5 $ implies the system is bounded. The 
velocity dispersion crossing time is $ t \approx 0.08 $, so we use a step size of 0.001, i.e. 
80 steps per crossing time.

\subsection*{Results}
%../pp-nbody/a.out cluster03.txt cluster03 1000 0.01 0.001 200 1
The video of this specific setup is \textit{simulations/cluster03.mp4}.
\subsubsection{Energy conservation and step sizes}



\subsection*{Binary Problem}
I am trying to conquer the problem of close encounters in our simulation.
The basic idea is to check if two stars 1) are close enough to each other and 2) $ K + V < 0 
$, every $ n $ steps. If yes, we just replace these two stars with one star at the center 
of mass. Binaries may further merge into trinaries, and so on so forth.


\subsubsection*{Realization of the Binary Problem}
If two stars both with velocity $ \sigma $ are in virial equilibrium, i.e. $ \alpha = K/|W| = 0.5 
$, the separation between them would be $ d_{\rm virial} = m /2 \sigma^2 = 0.016 $. 
However, the typical displacement of a particle in one step is $ d_{\rm step} = \sigma \Delta 
t \approx 1 $, much greater than $ d_{\rm virial} $. Therefore our simulation is not able to 
identify binary stars. We need a step size $ \sim $1000 times smaller to achieve the 
resolution of binary systems.

\subsubsection*{Searching for close encounters}
We consider two stars in the center-of-mass frame. We defined the following two 
parameters:
\begin{itemize}
	\item The \textit{close-encounter parameter} $ \alpha $ or $ \gamma $ which defines the 
	criteria of being close enough to each other:
	\begin{equation}\label{eq:r12}
	| \v{r}_1 - \v{r}_2 | < d_{\rm close} = \alpha \cdot d_{\rm virial} = \gamma \cdot d_{\rm step}.
	\end{equation}
%	where $ \alpha $ is a parameter to be determined.
	\item The \textit{escape parameter} $ \beta $ which confines the particles in a small 
	region:
	\begin{equation}\label{key}
	\frac{K}{|W|} < 1 - \beta^{-1}.
	\end{equation}
	This relation gives the largest separation between the two particles at any time, 
	\begin{equation}
	d_{\rm max} = \beta d_{\rm close},
	\end{equation}
	ignoring interactions with other particles
	\footnote{This relation is obtained by solving  equation $ (1 - \beta^{-1}) |V_0| + V_0 = 0 
	+ V_1 $ and $ V_0 \propto 1/d_0 $,	$ V_1 \propto 1/d_{\rm max} $.}.
	When $ d_{\rm max} \ll d_{\rm step} $ the two particles may be considered as a 
	binary.
\end{itemize}

The solution to the binary problem then becomes balancing the typical particle 
separation $ d_{\rm sepa} = L/\sqrt[3]{N} $, the one-step displacement $ d_{\rm 
step} = \sigma \Delta t $, the close-encounter parameter $ \alpha $ or $ \gamma $, and 
the escape parameter $ \beta $.

With $ d_{\rm sepa} \sim 5 $, we set $ \Delta t = 0.04 $, $ \alpha = 5 $ and $ \beta = 2 $, 
which imply $ d_{\rm step} = 1 $, $ d_{\rm close} = 0.08 $, and $ d_{\rm max} = 0.16 $.

\subsection{Results}
We artificially create a binary system and show that the binary is resolved with small step 
sizes but not resolved with large step sizes.

\begin{figure}
	\centering
	\caption{The tree code fails to resolve the binary system after a few orbits. Here $ 
	\epsilon = 10^{-5}, \Delta t = 10^{-5}, 4000 steps per output. $}
\end{figure}

%\subsection*{Morphology}
%We use disk galaxies in our simulation. Each disk galaxy has two components: a 
%thin disk and a thick disk. Each component has a density profile $ \rho(r, h) = 
%\rho_0 \, e^{-r/r_{\rm H}} \, e^{-h/h_{\rm H}} $. $ r $ and $ h $ are given by the 
%solution to the equation
%\begin{equation}\label{key}
%	x = 1 - (1 + \frac{r}{r_{\rm H}}) \, e^{-\frac{r}{r_{\rm H}}}
%\end{equation}
%where $ x $ is a uniform random number between 0 and 1.

%\subsection*{Disk Rotation}
%We use a uniform angular velocity $ \omega $.

%\subsection*{Setups}
%\begin{table}[h]
%	\centering
%	\begin{tabular}{ccccc}
%%		\hline{}
%		& $M/M_\odot$ & $r_{\rm H}$/kpc & $h_{\rm H} $/kpc & $\omega/?$ \\
%		\hline
%		thin disk & $5 \times 10^{10}$ & 3.5 & 0.3 & 1? \\
%		thick disk & $ 0.75 \times 10^{10} $ & 3.5 & 1.0 & 1? \\
%%		\hline
%	\end{tabular}
%	\caption{Parameters and errors from Lorentzian and Gaussian fits.}
%	\label{fit params}
%\end{table}
%
%\subsection{Units}
%[L] = kpc, [M] = $ 10^9 M_\odot $, [T] = 14.91 Myr.


\end{document}